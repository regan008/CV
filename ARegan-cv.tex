% Curriculum vitae

% Amanda Regan | aregan2@gmu.edu | http://amanda-regan.com

\documentclass[11pt]{article}
\usepackage{microtype}
\frenchspacing
\usepackage[letterpaper,margin=1in]{geometry}
\setlength\parindent{0in}
\setlength\parskip{6pt}
\setlength{\emergencystretch}{3em}
% \def\ind{\hangindent=1 true cm\hangafter=1 \noindent}
\usepackage[osf,sc]{mathpazo}
\usepackage{inconsolata}
\usepackage{fancyhdr}
\pagestyle{plain}
\frenchspacing
% \usepackage{multicol}
% \usepackage{graphicx}
% \usepackage[usenames,dvipsnames]{color}

% % ---- MARGIN YEARS
% \usepackage{marginnote}
% \newcommand{\years}[1]{\marginnote{#1}[4pt]}
% \renewcommand{\raggedleftmarginnote}{}
% \setlength{\marginparsep}{2pt}
% \reversemarginpar

% HEADINGS
\usepackage{sectsty}
\usepackage{titlesec}
\usepackage[normalem]{ulem}
\sectionfont{\rmfamily\mdseries\normal}
\subsectionfont{\rmfamily\mdseries\scshape\normal}
\subsubsectionfont{\rmfamily\bfseries\upshape\normalsize}
\titlespacing\section{0pt}{12pt plus 4pt minus 2pt}{0pt plus 2pt minus 2pt}
\titlespacing\subsection{0pt}{12pt plus 4pt minus 2pt}{0pt plus 2pt minus 2pt}
\titlespacing\subsubsection{0pt}{12pt plus 4pt minus 2pt}{0pt plus 2pt minus 2pt}
% disables chapter, section and subsection numbering
\setcounter{secnumdepth}{-1}
\usepackage[symbol]{footmisc}
\renewcommand{\thefootnote}{\fnsymbol{footnote}}
\providecommand{\tightlist}{%
  \setlength{\itemsep}{0pt}\setlength{\parskip}{0pt}}

% PDF SETUP
\PassOptionsToPackage{hyphens}{url}\usepackage[hidelinks, breaklinks, pdftex, pdftitle={Amanda E. Regan - CV}, pdfauthor={Amanda E. Regan}]{hyperref}

% DOCUMENT
\begin{document}
\thispagestyle{fancy}
\fancyfoot{}
\fancyhead{}
\renewcommand{\headrulewidth}{0pt}
%%% The next line will put the date revised in footer of first page
\rfoot{\scriptsize Current as of \today}

\hfill\hfill\hfill
\hfill\hfill\hfill
\hfill\hfill\hfill
\hfill\hfill\hfill
\begin{minipage}[t]{1.6in}
  \flushleft{\scriptsize  \texttt{\href{mailto:aeregan@clemson.edu}{aeregan@clemson.edu}}} \\
  \vspace{-0.065in}{\scriptsize  \url{https://amanda-regan.com}} \\
  \vspace{-0.065in}{\scriptsize  ORCID:
  \href{https://orcid.org/0000-0002-4260-5839}{\texttt{0000-0002-4260-5839}}}
\end{minipage}
\hfill
\begin{minipage}[t]{1.9in}
  \flushleft{{\scriptsize {Department of History and Geography} \\ Clemson University \\ 126D Hardin Hall \\ \vspace{-0.05in}
      Clemson, SC 29634}}
\end{minipage}


\vspace{0.1in}

{\Large Amanda E. Regan}\\[-0.1in]

\subsection{appointments}\label{current-position}

Assistant Professor, Department of History and Geography, Clemson Univeristy, (2022 - Present)

Lecturer, Department of History and Geography, Clemson University, (2021 - 2022)

Digital Humanities Postdoctoral Fellow, Center for Presidential History, Southern Methodist University, (2019 - 2021).

\subsection{education}\label{education}
\textbf{George Mason University}

\hfill\begin{minipage}{6.25in}

  PhD, History, 2019
  \vspace{0.10in}

\end{minipage}

\vspace{0.05in}

\textbf{California State University, San Marcos}

\hfill\begin{minipage}{6.25in}
MA, History, 2013

\vspace{0.10in}
BA, History, 2011


\end{minipage}


\subsection{peer reviewed publications}\label{peer-reviewed}

\emph{Shaping Up: Physical Fitness Initiatives for Women, 1880-1965}, (Under contract, University of Virginia Press).

Amanda Regan, Laura Crossley, and Josh Catalano, ``Grant Funded Research and Graduate Student Success,`` in \emph{Digital Futures of Graduate Study in the Humanities}, ed. Anouk Lang, Gabriel Hankins and Simon Appleford. (University of Minnesota Press, forthcoming.)

Amanda Regan and Eric Gonzaba, ``Mapping the `New Gay South:' Queer Space and Southern Life 1965-1980,'' \emph{The Southern Quarterly} 58, no. 1, (2020): 11-25. \url{muse.jhu.edu/article/868177}.

''Mining Mind and Body: Approaches and Considerations for Using Topic Modeling to Identify Discourses in Digitized Publications'', \emph{Journal of Sport History} 44, no. 2 (Summer 2017):160-77. \url{https://doi.org/10.5406/jsporthistory.44.2.0160}

\subsection{digital history projects}
``Mapping the Gay Guides,'' Co-Project Director and Digital Lead. \url{http://www.mappingthegayguides.org}.

``Mining Eleanor Roosevelt's \emph{My Day} Columns,'' with Joshua Catalano, May 2017. \url{https://regan008.shinyapps.io/mining_my_day/}.

``Mapping Gymnasiums in Boston,'' Visualization. \url{http://amanda-regan.com/bostongymnasiums/}.

``Twenty Years of RRCHNM History,'' Network Visualization, With Ken Albers, Peter Jones, Lincoln Mullen, Patrick Murray-John, Allison O’Connor, Faolan Cheslack-Postava, and Jannelle Legg, Fall 2014. \url{http://amanda-regan.com/RRCHNM20/}.

Roy Rosenzweig Center for History and New Media 20th Anniversary Archive, November 2014. With Jannelle Legg and Anne Ladyem McDivitt. \url{http://20.rrchnm.org/}.


\subsection{conferences and presentations}

``Geography, Media, and Queer Community Formation,`` Panelist, Critical Digital Humanities International Conference, September 2022.

``Mapping Historical LGBTQ Locations with Mapping the Gay Guides,`` Mapping and Data for Social Justice, Clemson Center for Geospatial Technologies, September 2022.

``Acquiring Fitness: Using Temporal Word Embeddings to Trace the Evolution of 'Fitness' in Physical Education Journals,`` Physical Cultures of the Body 2022, H.J. Lutcher Stark Center at the University of Texas Austin, January 2022.

``Mapping the Gay Guides: Using Digital History to Understand Queer Spaces in America,`` Association for Computing in the Humanities Conference, July 2021.

``Mapping the Gay Guides: Using Digital History to Explore LGBTQ Travel Guides, 1965-1980,`` with Eric Gonzaba, Southern Methodist Univeristy, November 2020.

``Shaping the `Soft American`': The Presidential Council on Physical Fitness 1955-1963,`` Tasting Freedom at George W. Bush Presidential Library and Museum, April 2021.

``Mapping the Gay Guides: Understanding Queer Spaces in Pre- and Post-Stonewall America`,` with Eric Gonzaba, DH2020, July 2020. \emph{[Cancelled due to COVID-19.]}

``Awash in a Sea of Content? Keep Up with the Field Using PressForward,`` with Joshua Catalano and Laura Crossley, Keystone DH, The Pennsylvania State University, July 2018.

``Mining Mind and Body: Approaches and Considerations for Using Topic Modeling to Identify Discourses in Digitized Publications,'' Doing Sport History in the Digital Present Workshop, Georgia Tech University, May 2016.

``Using the PressForward Plugin to Create Publications and Build Research Communities,'' with Amanda Morton, Advancing Research Communities and Scholarship Conference, Philadelphia, April 2015.

``Teaching Graduate Students to Code,'' contributor to poster session with Dr. Lincoln Mullen, American Historical Association conference, New York, January 2015.

\subsection{grants}
Co-Project Director, "Mapping the Gay Guides: Understanding Historical LGBTQ Spaces through Gay Travel Guides." Humanities Collections and Reference Resources grant from the National Endowment for the Humanities. (2021-2024) \$349,894.

Co-Primary Investigator, \emph{Mapping the Gay Guides.} California State University, Fullerton. Research, Scholarship, and Creative Activities Incentive Grant. \$14,715.

\subsection{reviews}\label{reviews}

Review of \emph{Governing Bodies: American Politics and the Shaping of the Modern Physique}, by Rachel Louise Moran, \emph{The Journal of Social History} 53, no. 3 (2019): 851-2. \url{https://doi.org/10.1093/jsh/shz006}.

Review of \emph{Body and Nation: The Global Realm of U.S. Body Politics in the Twentieth Century}, Emily S. Rosenberg and Shannon Fitzpatrick, ed., \emph{The Journal of Social History}, 49, no.2 (2015):455–456. \url{https://doi.org/10.1093/jsh/shv046}.

\subsection{academic and research positions}
Producer; \emph{The Past, The Promise, The Presidency Podcast: Season 1 Race and the American Legacy}; Center for Presidential History, Southern Methodist University, 2020-2021. \url{www.pastpromisepresidency.com}

Graduate Lecturer; Department of History and Art History, George Mason University, 2018-2019.

Graduate Research Assistant; Research Division; Roy Rosenzweig Center for History and New Media, George Mason University, August 2015-2018.

Software Development Manager; PressForward; Roy Rosenzweig Center for History and New Media, George Mason University, August 2016-2018. \url{http://www.pressforward.org}

Managing Editor; \emph{Digital Humanities Now}; Roy Rosenzweig Center for History and New Media, George Mason University, August 2016-2018. \url{http://www.digitalhumanitiesnow.org}

Digital History Fellowship; Roy Rosenzweig Center for History and New Media, George Mason University, August 2013-2015.

Producer; \emph{Digital Campus Podcast}; Roy Rosenzweig Center for History and New Media, George Mason University, August 2013-2015. \url{http://digitalcampus.tv/}

Digital and Institutional Review Board Consultant; \emph{War at Home and Abroad Digital: A Digital History Archive,} Department of History, California State University San Marcos, August 2012-January 2013, \url{http://www.csusmhistory.org/WAHA}.

\subsection{teaching and research fields}
U.S. History, U.S. Women's and Gender History, Digital History, Public History, Programming for Historians (computational history).

\subsection{teaching experience}

\textbf{Clemson University}
\begin{itemize}
  \item History 1010: United States History to 1877, Fall 2021 and Spring 2022.
  \item History 3050: United States History, 1917-1945, Fall 2022.
  \item History 3180: U.S. History, Fall 2022. 
  \item History 3890: Creative Inquiry - Mapping the Gay Guides, Spring 2022.
\end{itemize}
 
History 390: The Digital Past, George Mason University, 2017-2019.

History 121: U.S. History to 1865, North Virginia Community College, Summer 2018.

History 101: World Civilization to 1500, Teaching Assistant, California State University San Marcos, Spring 2013.

\subsection{guest lectures}
"\emph{Mapping the Gay Guides}" in Streaming Popular Feminisms (Honors First year seminar program), Clemson University, November 2022. 

"Algorithmic Bias" in Introduction to Sociology, Clemson University, April 2022.

"Data and Historical Sources: Building Mapping the Gay Guides" in Introduction to Museum Studies, York College of Pennsylvania, April 2021.

"Digital Public History" in Introduction to Public History, Clemson University, December 2020.

"Distant Reading in the Digital Humanities" in Introduction for Digital Humanities, Xavier University of Louisiana, October 2020.

"Doing Research in the Digital Age: Managing Your Personal Archive with Zotero and Tropy," Intro to Digital Humanities, Southern Methodist University, February 2020.

\subsection{digital humanities workshops}
Text Analysis with Voyant Tools, Collections As Data: Hack-to-Learn, Library of Congress, Washington D.C., May 2017.

Text Analysis for Digital Humanists, invited workshop and talk, University of California, Riverside, California, November 2016.

Graffiti Houses: The Civil War from the Perspective of Individual Soldiers; An NEH Landmarks in American History and Culture Workshop for Teachers, Graduate Research Assistant, George Mason University, June and July 2016.

Getting Started with PressForward, The Alfred P. Sloan PressForward Institute, 2015.

DC History Grad Zotero Workshop, with Jannelle Legg, George Mason University, October 2015.

DH Bridge: Encouraging Computational Thinking and Digital Skills in the Humanities, Coach, Roy Rosenzweig Center for History and New Media, George Mason University, October 2014.

\subsection{department, college, and university service}\label{Department, College, and University Service}
Member, Associate Dean for Undergraduate and Graduate Studies Search Committee, College of Architecture, Arts, and Humanities. Clemson University. Spring 2022.

\subsection{other professional and community service}\label{Other Professional and Community Service}
Member, Advisory Board, PressForward, (2022 - present).

Reviewer, Advanced Research Consortium (ARC), Fall 2022. 

Reviewer, National Endowment for the Humanities, Fall 2021 and Spring 2022.

Member, Advisory Board, Social Science Research Council, Research Area Mapping Platform project (2020 - present).

Reviewer, Digital Humanities 2020 Conference, Alliance of Digital Humanities Organizations, December 2019.

Participant, Manuscript workshop for Gregory Brew, Center for Presidential History, Southern Methodist University, November 2019.

Participant, Manuscript workshop for Lizzie Ingelson, Center for Presidential History, Southern Methodist University, October 2019.

Mentor, Diversity Outreach High School Program, Southern Methodist University, 2019-2020.

Participant, 2019 Mason Core assessment in IT and Computing, George Mason University, Spring 2019.

\subsection{other digital work}\label{other-digital-work}

Website redesign, \emph{PressForward}, January 2018, \url{http://www.pressforward.org}.

Website redesign, \emph{Appalachian Trail Histories}, January 2016. \url{http://appalachiantrailhistory.org/}.

Digital Design Consultant, \emph{Wearing Gay History: A Digital Archive of Historical LGBTQ T-Shirts}, July 2015. \url{http://www.wearinggayhistory.org}.

\subsection{awards and honors}
Honorable Mention, Garfinkel Prize in Digital Humanities for \emph{Mapping the Gay Guides}, 2020, Digital Humanities Caucus of the American Studies Association.

Emerging Open Scholarship Award for \emph{Mapping the Gay Guides}, 2021, Canadian Social Knowledge Institute.

Louise and Rudolf Fishel Graduate Fellowship, 2018, George Mason University, Fairfax, VA.

Joseph and Dorothy Censer Fellowship, 2016, George Mason University, Fairfax, VA.

Larry J. Hackman Research Residency, 2016, New York State Archives, Albany, NY.

Provost's Dissertation Completion Travel Grant, 2016, George Mason University, Fairfax, VA.

Passed Ph.D. Oral Comprehensive Exams with Distinction, George Mason University, Fairfax, VA.

Digital History Fellowship, 2013-2015, Roy Rosenzweig Center for History and New Media, George Mason University, Fairfax, VA.

Digital Humanities Summer Institute Tuition Scholarship 2014, 2015, University of Victoria, Victoria, B.C.

Distinguished Master's Thesis Award, 2013, California State University, San Marcos, Department of History, San Marcos CA.


\subsection{languages and digital humanities skills}
\textbf{Programming Languages:} R, PHP, JavaScript, Python

\textbf{Web Development Languages:} CSS, HTML

\textbf{Visualizations:} Shiny, D3.js, Leaflet

\textbf{Tools:} ArcGIS, CartoDB, GitHub, GoogleRefine, MALLET, QGIS

\textbf{CMS:} Omeka, WordPress

\subsection{professional memberships}
Organization of American Historians

American Historical Association

Association for Computing in the Humanities

Society for Historians of the Gilded Age and Progressive Era

\end{document}
